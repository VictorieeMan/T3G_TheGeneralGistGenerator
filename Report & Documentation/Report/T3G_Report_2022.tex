\documentclass[12pt,a4paper]{article}
	\setlength\parindent{0pt}
\usepackage[utf8]{inputenc}
\usepackage[english]{babel}
\usepackage{hyperref}
\usepackage{multicol}
\usepackage[yyyymmdd]{datetime}
	\renewcommand{\dateseparator}{--}
\usepackage{fancyhdr}
	\pagestyle{fancy}
	\fancyhead[l]{Stockholm}
	\fancyhead[c]{The General Gist Generator}
	\fancyhead[r]{\today}
	\fancyfoot[l]{}
	\fancyfoot[c]{\thepage}
	\fancyfoot[r]{}
\usepackage{amsmath}
\usepackage{amsfonts}
\usepackage{amssymb}
\usepackage{listings}
\usepackage{forest}
\usepackage{xcolor}
	\definecolor{codegreen}{rgb}{0,0.6,0}
	\definecolor{codegray}{rgb}{0.5,0.5,0.5}
	\definecolor{codepurple}{rgb}{0.58,0,0.82}
	\definecolor{backcolour}{rgb}{0.95,0.95,0.92}

\lstdefinestyle{mystyle}{
    backgroundcolor=\color{backcolour},   
    commentstyle=\color{codegreen},
    keywordstyle=\color{magenta},
    numberstyle=\tiny\color{codegray},
    stringstyle=\color{codepurple},
    basicstyle=\ttfamily\footnotesize,
    breakatwhitespace=false,         
    breaklines=true,                 
    captionpos=t,                    
    keepspaces=true,                 
    numbers=left,                    
    numbersep=5pt,                  
    showspaces=false,                
    showstringspaces=false,
    showtabs=false,                  
    tabsize=2
}
\lstset{style=mystyle}

%Example for code presentation
%\begin{lstlisting}[language=Python, caption=Python example]
%
%\end{lstlisting}

\title{The General Gist Generator}
\author{Victor Ekekrantz & Simon Jorstedt}
\date{\today}

\begin{document}
\thispagestyle{empty}
\begin{Huge}
The General Gist Generator
\end{Huge}
\begin{huge}
\\\\Victor Ekekrantz \& Simon Jorstedt
\end{huge}
\begin{LARGE}
\\\\Report last compiled: \today
\end{LARGE}

\section*{Forewords}
Is it possible to find a method of programmatically generate the gist of a comment section or the replies of a survey? This project will explore that line of questioning and aim to develop a method, capable of summarizing human written texts in a meaningful manner.
GitHub Repository: \url{https://github.com/VictorieeMan/T3G_TheGeneralGistGenerator.git}
\newpage
\tableofcontents
\section{File structure}
Readme for repository.
%
%\section{Program flow and Modules}
%
%\section{Code design, Algorithms \& Data structures}
%Comments on some of the design decisions within the program. 
%\subsection*{Error handling}
%Prevention is the best cure. This program attempts to catch errors before they become troublesome. Loading of modules are included in ''try--except'' blocks. And each of the functions in the \emph{user\_prompts.py} module, contain error handling and enforced integer input.
%
%\subsection*{Algorithms \& Data structures}
%The data structures in this program are kept simple. Only in use are the most fundamenal: \emph{integers}, \emph{floating points}, \emph{lists} and \emph{strings}. The most complex combination of these occur in the \emph{board} variable within the \emph{game\_tictactoe()} function from \emph{game\_sim.py}. The \emph{board} variable is constructed by the \emph{make\_\-move()} function, seen below, and it's a list of integer-lists. Where each integer-lists stores a player ID together with coordinates for a move on the board, this represents the placement of a piece.
%
%\begin{lstlisting}[language=Python, caption=from gs.check\_result()]
%# Parameterizes a line in the direction of adjacent
%# Then checks for three in a row along that line
%    for i in adjacent:
%        [dx,dy,dz] = [i[1]-x,i[2]-y,i[3]-z]
%        if dx != 0 or dy != 0 or dz != 0:
%            score = 0
%            for j in [0,1,2]: #If from endpoint
%                if [player,x+dx*j,y+dy*j,z+dz*j] in board:
%                    score += 1
%\end{lstlisting}

\end{document}